\documentclass{beamer}
\usepackage{newfile}
\usepackage{calc}

\title[]{title goes here}
\subtitle{what a lovely subtitle}

% speaker notes

% depends on newfile package
\newoutputstream{speakernotes}
\openoutputfile{\jobname.notes.xml}{speakernotes}

% necessary top bits
\addtostream{speakernotes}{<?xml version="1.0" encoding="UTF-8"?><notes>}

\usepackage{verbatim}

\newcounter{slide}

\makeatletter

% need the calc package because it works best to put the speakernotes
% env *after* a frame, and by then the page counter has been
% incremented. So we subtract 2 from that so that the note tags in the
% xml file correspond to the (0-based) page number in the PDF.

% if you're using some TeX thing where the speakernote environment is
% placed on the same page as the slide, you'd change the "-2" to "-1".
\newenvironment{speakernote}{%
 \@bsphack\setcounter{slide}{\thepage-2}%
\addtostream{speakernotes}{<note slide="\theslide">}%
 \let\do\@makeother\dospecials\catcode`\^^M\active
 \def\verbatim@processline{\addtostream{speakernotes}{\the\verbatim@line}}%
 \verbatim@start}%
{\addtostream{speakernotes}{</note>}\@esphack}
\makeatother

% close file when we're done
\AtEndDocument{\addtostream{speakernotes}{</notes>}\closeoutputstream{speakernotes}}

\begin{document}

\begin{frame}
  \titlepage
\end{frame}
\begin{speakernote}
title page note
\end{speakernote}

\begin{frame} \frametitle{first frame with regular content}
here's stuff on page 1
\end{frame}
\begin{speakernote}
oh my

some notes. yes, this is verbatim: \foo # \\ $ 
 TeX doesn't mind ampersands here, but bare ampersands aren't valid XML
-- use "&amp;"

 same for bare less-than signs: use &lt;. 
\end{speakernote}
% $

\begin{frame} \frametitle{xyz}
lorem ipsum

\pause

some text for the second part of the frame
\end{frame}
\begin{speakernote}
  some stuff after the frame with two overlays

btw, utf-8 text should be okay: λ ⇒ ∞ ↔

and entity references are ok too: &#955;
\end{speakernote}

\begin{frame}
  just a regular ol' frame
\end{frame}
\begin{speakernote}
  just a regular ol' note
\end{speakernote}

\begin{frame}
  one
  \pause
  two
  \pause
  three
\end{frame}
\begin{speakernote}
  something about the frame with three overlays
\end{speakernote}

\end{document}

%%% Local Variables: 
%%% mode: latex
%%% TeX-PDF-mode: t
%%% End: 
